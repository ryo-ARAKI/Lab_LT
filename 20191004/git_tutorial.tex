\documentclass[12pt,dvipdfmx,svgnames,uplatex,aspectratio=169]{beamer}
% \documentclass[12pt,dvipdfmx,svgnames,uplatex,aspectratio=169,handout]{beamer}
%
% ===========================================
% 図・表関係
% ===========================================
% \usepackage[draft]{graphicx}
\usepackage{graphicx}
\graphicspath{{./pics/}}  % \includegraphicsで参照するディレクトリ
%
% ===========================================
% 参考文献
% ===========================================
\usepackage[url=false,isbn=false,doi=false]{biblatex}
% \addbibresource{./bib_textbooks.bib}  % 教科書など
% \addbibresource{./bib_articles.bib}  % 論文など
%
% ===========================================
% 独自スタイルの導入
% ===========================================
\usepackage{/home/ryo/.config/LaTeX/mystyle_beamer}
\newcommand{\git}[1]{{\colorbox{WhiteSmoke}{\texttt{#1}}}}  % gitコマンドを薄い灰色の背景とtypewriterフォントで表記
%
% ===========================================
% 表紙の記述
% ===========================================
\title{非情報系理系院生のための \\
  モダンな開発環境づくり}
\subtitle{その1. Git/GitHubを使ったソースコード管理}
\author{荒木 亮}
\institute{阪大院基礎工・後藤研}
\date{\today}

% +++++++++++++++++++++++++++++++++++++++++++
% 本文
% +++++++++++++++++++++++++++++++++++++++++++

\begin{document}
\frame{\maketitle}
\begin{frame}{もくじ}
  \tableofcontents
\end{frame}

% ===========================================
% 以下,スライドを記述する
% ===========================================
\section{目標}
\begin{frame}{\insertsection}
  \begin{screen}
    \centering
    ソースコードやLaTeXのバージョンを,Git/GitHubで管理する
  \end{screen}

  \begin{itemize}
    \item Git初心者が,「ファイル名に日付をつけてバックアップ」\\をGitで代替できるようになることをめざす
  \end{itemize}

  \begin{alertblock}{説明しないこと}
    \begin{itemize}
      \item \git{branch} を使った同時並行的な開発
      \item \git{pull request} を使ったチームでの開発
      \item その他色々ややこしいコマンド
    \end{itemize}
  \end{alertblock}
\end{frame}

\section{Gitの構造}
\begin{frame}{\insertsection}
  \begin{screen}
    \begin{itemize}
      \item 作業ツリー
      \item ステージングエリア
      \item ローカルリポジトリ
      \item リモートリポジトリ
    \end{itemize}
    を表す図を追加
  \end{screen}
\end{frame}

\begin{frame}{\insertsection}
  \begin{columns}[T] % 上辺をあわせる
    \begin{column}{0.5\textwidth}
      \begin{block}{作業ツリー}
        block
      \end{block}
      \begin{block}{ステージングエリア}
        block
      \end{block}
    \end{column}
    \begin{column}{0.5\textwidth}
      \begin{block}{ローカルリポジトリ}
        block
      \end{block}
      \begin{block}{リモートリポジトリ}
        block
      \end{block}
    \end{column}
  \end{columns}
\end{frame}

\section{Gitのコマンド}
\begin{frame}{\insertsection}
  \begin{columns}[T] % 上辺をあわせる
    \begin{column}{0.5\textwidth}
      \begin{block}{\git{git add}}
        block
      \end{block}
      \begin{block}{\git{git commit}}
        block
      \end{block}
    \end{column}
    \begin{column}{0.5\textwidth}
      \begin{block}{\git{git push}}
        block
      \end{block}
      \begin{block}{\git{git pull}}
        block
      \end{block}
    \end{column}
  \end{columns}
\end{frame}

\section{分散型バージョン管理}
\begin{frame}{なんでこんな難しいんや:\insertsection}
  \begin{columns}[c] % 中央をあわせる
    \begin{column}{0.5\textwidth}
      \begin{screen}
        クラウドサービスとGitの\\差異を表す図
      \end{screen}
    \end{column}
    \begin{column}{0.5\textwidth}
      \begin{block}{commit}
        block
      \end{block}
    \end{column}
  \end{columns}
\end{frame}

\section{なぜGitを使うのか}
\begin{frame}{\insertsection}
  \texttt{git\_tutorial.md}から資料を作成する
\end{frame}

\section{今日から始めるGit生活}
\begin{frame}{\insertsection}
  \begin{enumerate}
    \item GitHubでアカウント作成
    \item Education planの作成
    \item 研究用リポジトリの作成
    \item コードを保存しているディレクトリを登録
    \item ファイル編集\(\to\) \git{add} \(\to\) \git{commit} \(\to\) \git{push}
    \begin{itemize}
      \item[※] 見返してわかりやすい「commitメッセージ」をつける
    \end{itemize}
    \item GitHubのページを確認し,コードが変更されていることを確認
  \end{enumerate}
\end{frame}

\section{便利なGitコマンド}
\begin{frame}{\insertsection}
  \begin{columns}[T] % 上辺をあわせる
    \begin{column}{0.5\textwidth}
      \begin{block}{\git{git checkout .}}
        block
      \end{block}
      \begin{block}{\git{git stash}}
        block
      \end{block}
    \end{column}
    \begin{column}{0.5\textwidth}
      \begin{block}{\git{git commit --amend}}
        block
      \end{block}
      \begin{block}{\git{git reflog}}
        block
      \end{block}
    \end{column}
  \end{columns}
  その他, \git{git revert} \git{git reset}

  gitコマンドへのオプション,\texttt{.gitignore}
\end{frame}

\section{リンク集}
\begin{frame}{\insertsection}
  \texttt{git\_tutorial.md}から資料を作成する
\end{frame}

\end{document}
